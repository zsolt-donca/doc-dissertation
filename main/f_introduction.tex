
\chapter{Introduction}\label{chap:intro}

\section{Abstract}\label{sec:abstract}

FEST-Logging an extension for the functional testing system FEST-Swing that tests Swing based desktop Java applications. The extension is a framework that easily integrates into existing FEST-Swing-based test systems requiring few code changes. The only requirements is that they have to be built on a common design pattern.

\section{Overview}\label{sec:overview}

This document consists of the following parts:
\begin{itemize}
\item a brief explanation of the test framework (\fullref{chap:fest-swing})
\item a common design pattern associated with UI tests (\fullref{chap:user-actions-adapters})
\item the features of the extension framework
	\begin{itemize}
	\item \fullref{chap:test-auditing}
	\item \fullref{chap:edt-dispatch}
	\item \fullref{chap:simulated-dnd}
	\end{itemize}
\item \fullref{chap:nfr}
\end{itemize}

\noindent The main features of the extension framework are the following:
\begin{itemize}
\item test auditing (\fullref{chap:test-auditing}):
	\begin{itemize}
	\item the goal is to ease the troubleshooting of non-independent tests by providing de-tails of the test runs
	\item intended to log runtime data (e.g. arguments of the method calls)
	\item seamlessly integrates into the various layers of the tests (i.e. test, user action, adapter, fixture, driver, robot)
	\item allows to create a movie for a test (e.g. creating screenshots after each user action of a test)
	\item reports the audited data in easy-to-use output formats (e.g. interactive HTML, with filtering and sorting)
	\end{itemize}
\end{itemize}

\begin{itemize}
\item automatic thread dispatch (\fullref{chap:edt-dispatch}):
	\begin{itemize}
	\item the goal is to simplify the development of thread-safe tests by automatically dele-gating operations to Swing EDT
	\item the operations need to be marked with certain annotations and the runtime sys-tem behind the scenes automatically handles the delegation
	\end{itemize}

\item simulated drag-and-drop (\fullref{chap:simulated-dnd}):
	\begin{itemize}
	\item the goal is to solve a major problem for Continuous Integration systems
	\end{itemize}
\end{itemize}

\noindent The implementation details are the following:

\begin{itemize}
\item Scala: as a language of choice for implementing the framework, while keeping full compati-bility with Java systems;
\item Spring: configuration, IoC and AOP capabilities;
\item AspectJ: aspects implementing the auditing (e.g. by altering the test methods, user actions etc.)
\item Mockito: while primarily used for testing (for mocking dependencies), also can be used to mock certain classes of the AWT drag-and-drop system, with the purpose of simulated drag-and-drop by programmatically triggering data export and import.
\end{itemize}
