% BE�LL�T�SOK - JOBB NEM V�LTOZTATNI
\documentclass[final]{ubb_dolgozat}
\usepackage{definitions}
%%


% milyen nyelveken akarunk forr�sk�dot megjelen�teni
\lstloadlanguages{java,scala}

% ezt be lehet tenni MINDEGYIK megjelen�tend� k�d el�.
\lstset{language=scala}


%%%%%%%%%%%%%%%%%%%%%%%%%%%%%%%%%%%%%%%%%%%%%%%
%%!!          EZT KELL V�LTOZTATNI       !!%%%%
%%     A DOLGOZAT C�MOLDAL�NAK ELEMEI        %%

%% MELYIK �VBEN ADJUK LE
\submityear{%
2014
}
%% MELYIK H�NAPBAN ADJUK LE
\submitmonthHU{%
J�lius
}
\submitmonthRO{%
Iulie
}
\submitmonthEN{%
July
}

\titleHU{%
Egy �gyes Vizsgadolgozat
}

% Amennyiben sz�ks�ges, az al�bbi sorokat ki kell komment-ezni �s 
% be�rni a megfelel� c�meket

\titleEN{%
  FEST-Logging
}

\titleRO{%
  FEST-Logging
}

\author{%
Zsolt Donca
}

%%
\tutorHU{%
dr. Csat� Lehel, egyetemi docens\\
%{\large Babe\c{s}--Bolyai Tudom�nyegyetem,\\
% Matematika �s Informatika Kar}% ha k�l�nb�zik, akkor fel kell t�ntetni
}
%%
\tutorRO{%
Conf. dr, Lehel Csat�\\
% {\large Universitatea Babe\c{s}--Bolyai,\\ % dac� difer�!!!
% Facultatea de Matematic\u{a} \c{s}i Informatic\u{a} }%
}
%%
\tutorEN{%
Assoc. prof. dr. Lehel Csat�
% {\large Babe\c{s}--Bolyai University,\\
% Faculty of Mathematics and Informatics}
}



%\usepackage{t1enc}
%\usepackage{ucs}
%\usepackage[utf8x]{inputenc}

%\usepackage[pdftex]{graphicx}
%\usepackage{url}
%\usepackage{algorithmic}
%\usepackage{amsmath}
%\usepackage{amssymb}

%\usepackage[table]{xcolor}
%\usepackage{listings}
%\usepackage[colorlinks=true]{hyperref}
\newcommand*{\fullref}[1]{\hyperref[{#1}]{\autoref*{#1} \nameref*{#1}}} % One single link

\lstdefinelanguage{scala}{
  morekeywords={abstract,case,catch,class,def,%
    do,else,extends,false,final,finally,%
    for,if,implicit,import,match,mixin,%
    new,null,object,override,package,%
    private,protected,requires,return,sealed,%
    super,this,throw,trait,true,try,%
    type,val,var,while,with,yield},
  otherkeywords={=>,<-,<\%,<:,>:,\#,@},
  sensitive=true,
  morecomment=[l]{//},
  morecomment=[n]{/*}{*/},
  morestring=[b]",
  morestring=[b]',
  morestring=[b]"""
}

\usepackage{parskip}

\usepackage{color}
\definecolor{dkgreen}{rgb}{0,0.6,0}
\definecolor{gray}{rgb}{0.5,0.5,0.5}
\definecolor{mauve}{rgb}{0.58,0,0.82}
 
% Default settings for code listings
\lstset{%frame=tb,
  language=scala,
  aboveskip=3mm,
  belowskip=3mm,
  showstringspaces=false,
  columns=flexible,
  basicstyle={\small\ttfamily},
  numbers=none,
  numberstyle=\tiny\color{gray},
  keywordstyle=\color{blue},
  commentstyle=\color{dkgreen},
  stringstyle=\color{mauve},
%  frame=single,
  breaklines=true,
  breakatwhitespace=true
  tabsize=3
}



%\setcounter{secnumdepth}{3}
%\setcounter{tocdepth}{3}

% experimentary tabular stuff

\usepackage{array}
\newcolumntype{L}[1]{>{\raggedright\let\newline\\\arraybackslash\hspace{0pt}}m{#1}}
\newcolumntype{C}[1]{>{\centering\let\newline\\\arraybackslash\hspace{0pt}}m{#1}}
\newcolumntype{R}[1]{>{\raggedleft\let\newline\\\arraybackslash\hspace{0pt}}m{#1}}



\begin{document}

%% ABSTRAKT
\begin{abstractEN} % ANGOL V�LTOZAT

% a lenti r�szt �rtelemszer�en ki kell t�lteni a dolgozat angol kivonat�val.
% A BEGIN ... END k�z�tt CSAK A SAJ�T SZ�VEG kell, hogy legyen.
% Az utols� mondatot benne kell hagyni, mely �ltal kijelentitek, hogy a munk�tok SAJ�T.


{ \color{gray!60!red}
	FEST-Logging an extension for the functional GUI testing system FEST-Swing that tests Swing based desktop Java applications or applets. FEST-Logging logs execution details of test suites, helping development, troubleshooting or optimization efforts, easily integrating into existing test systems requiring few code changes.

FEST-Logging is demonstrated with a simple file manager application developed in the functional programming language \emph{Scala}. The application was developed using the library \emph{Scala.React} as a proof-of-concept of the article \emph{Deprecating the Observer Pattern with Scala.React}\cite{DeprecatingObservers}.

	\vfill
  
}

This work is the result of my own activity. I have neither given nor received unauthorized assistance on this work.

\end{abstractEN}


\maketitle

{ \baselineskip 1ex
  \parskip 1ex
  \tableofcontents
}



\chapter{Introduction}\label{chap:intro}

Interactive applications are in a continuously increasing demand, driven by the expansion of computing and mobile devices. A reactive application needs to be ``readily responsive to a stimulus'', thus needs to:
\begin{enumerate}
\item \textbf{react to events} --- the application needs to have an \emph{event-driven} design;
\item \textbf{react to load} --- the application needs to scale well as the amount of interaction and data increases;
\item \textbf{react to failure} --- the application must be resilient to error and needs to recover at all levels;
\item \textbf{react to users} --- the application must be responsive to the user, regardless of load.
\end{enumerate}
There is a growing trend in the programming community over the importance of reactive programming. \cite{ReactiveManifesto}

Development of interactive applications is difficult engineering task: the continuous user input and output, the high number of possible internal states makes it difficult to achieve the above desired properties. The event-driven nature of the applications tend to have a negative effect on the maintainability: the state machines and the inversion of control that is needed to deal with events increase the application's complexity, resulting in code that is possibly difficult to maintain, develop, and is error-prone. 

The Scala library \emph{Scala.React}, developed together with the paper \emph{Deprecating the Observer Pattern with Scala.React} offers a solution to the above problem. Its goal is to introduce multiple reactive programming abstractions enabling an event-driven design using declarative implementations, and together with a high-order data-flow DSL\footnote{Domain-Specific Language} embedded into \emph{Scala} makes it possible to implement complex multi-event logic in a direct imperative style, without the need for inversion of control or to implement state machines. \cite{DeprecatingObservers}

\section{Scala Commander}

We implement a simple desktop application with the goal of demonstrating the capabilities of \emph{Scala.React} in a practical example. The application is \emph{Scala Commander}, a file manager application with a user interface that constantly reacts to user interaction. The reactive abstractions in \emph{Scala.React} enabled the design of the application to consist of simple ``building blocks'' called \emph{signals} to be wired together representing the internal state of the application. The relationship between signals is declarative, and there is no development effort needed for change propagation or to maintain consistency. The logic of the application is implemented using \emph{reactors}, with a data-flow DSL that lets us write logic dealing with multiple events in a direct pseudo-code-like style, significantly reducing the complexity of the code.

\section{FEST-Logging}

Testing of applications user interfaces is also important. The library \emph{FEST-Swing} offers an easy and intuitive way to write tests for \emph{Swing} UIs that are compact, easy to write and read like a specification. However, dealing with large number of tests can be problematic: shared state between individual tests can affect the stability of the tests. 

We also implement \emph{FEST-Logging}, the library extending \emph{FEST-Swing} with the goal of easing the testing of Swing user interfaces. FEST-Logging is an agent running along test suites, gathering information while test run providing insight into what happens in the test suites, helping maintenance and troubleshooting of failed tests. FEST-Logging gathers information on the tests such as total duration and what actions did the test execute (along with the actual \emph{arguments}). It also takes screenshots of failed tests, and optionally can create a ``movie'' of the tests by taking screenshot after each test action.

To ease development of tests, \emph{FEST-Logging} is integrated with \emph{Cacio-tta}, a graphics stack that enables the tests to run in background. There is also a simple annotation-based feature that provides automatic delegation of UI-related operations to the Swing EDT thread.

\section{The structure of this paper}

In chapter \ref{chap:theory}, we present the theoretical background necessary to understand the applications \emph{Scala Commander} and \emph{FEST-Logging}:
\begin{itemize}
\item In section \ref{sec:theory_scala} we give a generic overview of Scala, the object-functional programming language. Our goal is to give a quick overview of the language constructs in Scala, and to present the most important features that the concepts in \emph{Scala.React} or in \emph{Scala Commander} rely on. % \emph{Scala.React} and \emph{Scala Commander} in particular is made possible by the highly extensible and feature-rich nature of \emph{Scala}, the object-functional programming language.

\item In section \ref{sec:theory_scala-swing} we present \emph{Scala-Swing}, a Scala library built on \emph{Java Swing}, the primary Java GUI widget toolkit. \cite{SwingWiki} Scala Commander uses Scala-Swing to implement the user interface.

\item In section \ref{sec:theory_scala-cps-plugin}, we present the \emph{Scala CPS plugin}, a compiler plugin providing delimited continuations using a continuation passing style transformation. These delimited continuations make it possible for the Scala.React data-flow DSL to suspend and resume the control flow in the reactors.

\item In section \ref{sec:theory_scala-react}, we present the library \emph{Scala.React}, its main concepts and some implementation details.

\item In section \ref{sec:theory_fest-swing}, we present the library \emph{FEST-Swing} that provides functional testing for Swing user interfaces.

\item In section \ref{sec:theory_design-patterns}, we present the Model-View-Controller design pattern that Scala Commander is based on.

\item In section \ref{sec:theory_functional-ui-testing}, we present a design pattern commonly associated with testing of user interfaces.
\end{itemize}

In chapter \ref{chap:implementation-details}, we present the implementation details of the two main applications developed for this paper:
\begin{itemize}
\item In section \ref{chap:impl_scala-commander}, we present \emph{Scala Commander}, the file manager application demonstrating the capabilities of \emph{Scala.React}.
\item In section \ref{chap:impl_fest-logging}, we present \emph{FEST-Logging}, the library extending \emph{FEST-Swing} with the goal of easing the testing of Swing user interfaces.
\end{itemize}









\chapter{Theoretical background}\label{chap:theory}

\section{FEST-Swing}\label{sec:fest-swing}

FEST (Fixtures for Easy Software Testing) is a collection of libraries whose mission is to simplify software testing. It is composed of various modules, which can be used with TestNG or JUnit. The most significant module is the FEST-Swing module.

The Swing module provides a simple and intuitive API for functional testing of Swing user interfaces, resulting in tests that are compact, easy to write, and read like a specification. Tests written using FEST-Swing are also robust. FEST simulates actual user gestures at the operating system level, ensuring that the application will behave correctly in front of the user. It also provides a reliable mechanism for GUI component lookup that ensures that changes in the GUI's layout or look-and-feel will not break the tests \cite{FESTMain}.

\subsection{Introduction}

A FEST-Swing test is testing either individual frames (the building blocks of a UI), or entire Swing applications or Applets. FEST simulates actual user gestures (mouse movements, keys presses) at the operating system level, ensuring that during the application will behave in the same way as in the front of the user. It uses AWT Robot, which generates events in the platform's native input queue. Thus, the test needs to create the components and make them visible on the screen, and the tests will actually moves the mouse cursor, and not just only generates mouse move events.

\subsection{Architecture}

FEST Swing's component fixture architecture is separated into several layers (from bottom to top):
\begin{enumerate}
\item BasicRobot: Simulates a user interacting with a mouse and keyboard. It uses the AWT Robot to generate native input events.
\item Component driver: This layer does all the ``heavy lifting.'' All interaction with a GUI component is done in this layer. It knows how to simulate events and check the state of a specific GUI component. For example, JComboBoxDriver knows how to simulate a user using a JComboBox (selecting a particular element) and how to verify the state of it (which element should be selected.)
\item Component fixture: This layer sits on top of the driver. It provides a fluent interface to that and makes the API easier to write and read. Users of FEST write their GUI tests using fixtures, not drivers. There is one fixture per Swing component, and each fixture has the same name as the Swing component they can handle ending with ``Fixture.'' For example, a JButtonFixture knows how to simulate user interaction and verify state of a JButton. Fixtures can be considered the ``user interface'' of the FEST-Swing library.
\end{enumerate}

The architecture also supports extensions, so developers of application using custom Swing components can write their own fixtures and component drivers.

\subsection{Example}

Writers of GUI tests need to use the fixtures located in the package org.fest.swing.fixture. These fixtures provide specific methods to simulate user interaction with a GUI component and they provide assertion methods that verify the state of the GUI component. Although it is possible to work with the FEST BasicRobot directly, the BasicRobot is too low-level and requires considerably more code than the fixtures.

As a concrete example, let us consider a very simple JFrame that contains a JTextField, a JLabel and a JButton, and component has its unique name. The expected behavior of this GUI is the following: when user clicks on the JButton, the text of the JTextField should be copied to the JLabel. (Figure \ref{fig:example_jframe}) 

\begin{figure}[h!] \label{fig:example_jframe}  
  \centering
    \includegraphics[width=0.6\textwidth]{images/example_jframe.png}
  \caption{A very simple JFrame that contains a JTextField, a JLabel and a JButton.}
\end{figure}

To the frame, the test's setup method needs to create the frame (in the EDT, delegating it with GuiActionRunner), create a fixture for it, and make it visible (Figure \ref{fig:example_setup_method}).

\begin{figure}[h!] \label{fig:example_setup_method}
\begin{lstlisting}
protected def onSetUp() {
    val frame: MyFrame = GuiActionRunner.execute(new GuiQuery[MyFrame] {
        protected def executeInEDT : MyFrame = {
            return new MyFrame
        }
    })
    window = new FrameFixture(robot, frame)
    window.show() // shows the frame to test
}
\end{lstlisting}
\caption{The setup method creating the frame and the fixture, and making it visible.}
\end{figure}

The test method can use the fixture to simulate a user interacting with a GUI in order to verify that such GUI behaves as we expect. The user interactions and the assertions in the test are fluent and simple. The method looks up the UI components by their unique names. (Figure \ref{fig:example_test_method})

\begin{figure}[h!] \label{fig:example_test_method}
\begin{lstlisting}
@Test 
def shouldCopyTextInLabelWhenClickingButton {
    window.textBox("textToCopy").enterText("Some random text")
    window.button("copyButton").click()
    window.label("copiedText").requireText("Some random text")
}
\end{lstlisting}
\caption{The test method.}
\end{figure}

It is important to mention that besides testing individual components that build up an application, FEST-Swing also supports testing entire applications. In such cases, the first tests starts up the application with ApplicationLauncher that understands how to launch an application using its main method. Once the application is started, the test just needs to find the application's main window.

\subsection{Threading model}

The documentation of FEST-Swing strongly advises to respect Swing's threading rules \cite{OracleSwingThreading} both in the tests and in the tested application (this remains only an advice because Swing itself does not enforce thread safety). In short, the cardinal rule is the following: creation and access (both read and write) of Swing components should be done in the Event Dispatch Thread (EDT.) Since JUnit and TestNG tests do not run on the EDT, creation and any direct access in the tests to Swing components should be delegated to EDT via the utility class GuiActionRunner (or via other tools).

To ensure that the threading rules are respected throughout the tests and the tested application, FEST Swing provides the class FailOnThreadViolationRepaintManager. It forces a test failure if access to Swing components is not performed on the EDT. However, it only detects component creation and writing, and is unable to detect read operations (component creation and writing triggers repaint events, but read does not).

\subsection{Limitations}\label{sec:fest-swing-limitations}

Because FEST-Swing is actually simulating user interaction, it needs an environment very similar to actual user environments: the tested application needs to be active and in focus, and needs to move the mouse. Therefore, that machine that the tests run on cannot be used for the entire duration of the tests. In addition, the tests probably fail whenever another applications unexpectedly gets the focus and covers the application's window (the mouse no longer clicks the correct application). This can become a burden because UI tests, by their nature, can take a significant amount of time to execute and especially when using agile development methods, it is preferable to run them frequently.

Because of this, it is common to delegate the tests to Continuous Integration (CI) systems (e.g. Hudson/Jenkins, TeamCity etc.) When the CI platform is based Linux, BSD or UNIX-style operating systems, it is common that the server does not even have the X Window system, running applications with UI impossible. For this case, Xvfb offers a simple and straightforward solution \cite{FESTxvfb}.

However, if the target platform is Windows, there are a whole set of issues, and the known solutions are problematic, and are very complicated and time consuming to setup. \cite{HudsonUnderWindows} \cite{Cacio_Tta_FEST}

To overcome all of these limitations, OpenJDK's project Caciocavallo can be used \cite{Cacio_Tta_FEST} to provide a graphics stack for the Java VM that is completely independent from the environment, eliminating any need for platform-dependent setup for CI systems. It renders everything into a virtual screen (which is simply a BufferedImage object), and is driven solely by AWT Robot events. However, this also has its own limitations; the most important being that drag-and-drop \cite{IntroDnD} is unsupported, thus it needs certain workarounds (see \fullref{sec:simulated-dnd}).

\section{User Action and Adapter design pattern}\label{sec:user-actions-adapters}

The users using an application with a UI typically interact with it by executing a series of operations that change the application's state. These operations might themselves be consisting of multiple operations of the UI components, changing the UI's state and, indirectly, changing the application's internal state. UI test simulating the users can incorporate in their architecture the concept of separation: the tests can interact with a user action layer (consisting of the actions the user can do with the application), and the user actions can interact with an adapter layer that itself interacts with the UI components (e.g. by using FEST-Swing).

The same separation applies to the tests' assertions: the tests assert the application's state through the user actions, and the user actions assert the UI's state, and thus, indirectly, assert the application's internal state.

\subsection{Adapters}

The adapters understand how to execute operations on the application by using the UI components. 

For example, to create a new text file in an editor application supporting multiple types of files, the user typically needs to click on the File menu, then on the menu item New, and then on the sub-menu-item Text file. The combinations are usually limited (the submenu-item cannot be clicked without clicking the menu item first), and sometimes require some checking (if the File menu is already visible for whatever reason, the test should not click it again because that would only close the menu).

These operations come from the functional specification of the application, and their implementation details are only dependent on the UI design. These operations can be structured into adapters, where each logically bound component group can have a corresponding Adapter class (e.g. FileMenuAdapter), where the methods are the operations (e.g. FileMenuAdapter\#clickNewTextFile()). The adapter class can have a fixture for each Swing component, and the methods can directly interact with them.

This approach also leads to tests that are more resistant to UI changes, requiring less code change for each UI design change. For example, if the in the UI design of the above text editor the Text file menu item is moved directly into the File menu under the name New text file, only the appropriate adapter class needs to be modified, while all the tests using this operation are left unchanged. Whereas, if the code using the fixtures and clicking on the menu items was present in all the tests, such a change would need all the tests to be changed.

\subsection {User actions}

The user actions understand how to execute operations of the application by using the adapters. 

For example, let us consider a case where the user wants to save the file in the current editor of an editor application with a new name. The user needs to click the Save as menu item, resulting in the appearance of a browser dialog. After that, he needs to enter the new file name into a text box, and then needs to click the OK button of the dialog, and finally wait for the dialog to disappear. In this case, the file menu can have an adapter; the file browser dialog can have an adapter; and using both of them in the same time in a given way can represent a user action. For example, the user actions related to the current edited document can be grouped into a user action class (EditorUserAction), and where the methods are the user actions (EditorUserAction.saveAs(newFileName)).

\section{Scala-React}\label{sec:scala-react}

Programming systems with interactive user interfaces requires a considerable amount of engineering to deal with continuous user input and output, yet the predominant programming models dealing with continuous state change in an application are relatively simplistic, mainly consisting of the observer pattern. The Scala library \emph{Scala-React} offers a reactive data-flow programming model, and by that it solves numerous problems that a design based on the observer pattern faces.\cite{DeprecatingObservers} Amongst multiple Scala features, this is enabled by the higher order concepts of events and state, enabled by Scala's higher-order functions, and by a data-flow DSL embedded into Scala, enabled by continuation passing style transformation that \emph{Scala} offers.






































\chapter{Implementation details}\label{chap:implementation-details}

\section{Test auditing}\label{sec:test-auditing}

\subsection {Non-independent tests}

Test suites testing even relatively complex applications can easily contain some hundred tests. Because UI tests, by their nature, typically take much time, the total run time of an entire test suite is usually important. In order to keep the overall runtime at minimum, it is typical to avoid the creation, initialization and the destruction of the individual UI components for each test (which might also be difficult or impossible depending on the architecture of the application). A possible option is to test entire applications and to reuse the same components between the tests.

The problem with testing entire applications throughout entire test suites is that the tests can become dependent on the internal state of the application and on the state of the UI. Thus, it is possible for one test to affect the outcome of another, e.g. one failing test can leave the application in such a state that no other further tests will pass. Having non-independent tests can lead to very fragile and unstable test systems and thus should be avoided as much as possible.

Unstable test system can very difficult to cope with. Tests can seemingly fail randomly, and it can be very difficult to reproduce failures that originate from typically a subtle change in the application's state that was made by a previous test (possibly executed multiple tests before). FEST supports saving screenshots anytime during a test, but the test must explicitly save it and it is usually done only on errors. Since in the case of unstable tests, the errors typically manifest themselves only in the tests that are affected by the erroneous tests’ side effects, saving screenshots for failing tests does not help much on its own.

Understanding the data flow of all the tests of an entire test suite, knowing all the operations that were made on the application’s state, along with screenshots made after all important steps can greatly reduce the debugging efforts of an unstable test suite and can benefit the test development and maintenance process.

\subsection {AspectJ-based auditing}

A possible implementation of the auditing is by using AspectJ. The runtime system could rely on using join points defined by automatically generated pointcuts, and by using advices wrapping around the join points, the entire method invocation context (stack trace, method arguments) can be stored (logged). \cite{AOPwiki} From the context of each method invocation, an execution tree can then be built and visualized.

The pointcuts could be generated for the methods of the classed in the test, user action, adapter, fixture, driver or robot layers. The deeper auditing goes through the layers, the more runtime data is gathered and the bigger is the performance cost of the auditing. Thus, it is important to make the set of the audited layers and methods configurable.

Screenshots could be similarly generated after each user action (or adapter operation) of a test.

\subsection {Formatting the audited data}

Since the runtime data set can be very huge for long tests runs, it is important to provide views that presents the data in an easily accessible, intuitive and adaptive form. 

A possible implementation is to generate an HTML page that contains all the data in the form of a pivot table, with a tree axis showing the call hierarchy, and the columns presenting the method arguments and other context info. The pivot table would support drill-down in the call hierarchy. This table would support the following:
\begin{itemize}
\item Closing and expanding nodes (method calls) in the call hierarchy;
\item Filtering and sorting by the context data (e.g. method names, arguments, annotations).
\end{itemize}

The following is a simple example of a test of a text editor application consisting of the following steps:
\begin{itemize}
\item create a new text file
\item enter the text “hello” and “ world”, respectively
\item save the text file to the disk
\item assert that the editor shows the file name in the window’s title
\item assert that the file exists on the disk and has the appropriate content
\end{itemize}

Table \ref{fig:formatted_audited_data_report} shows the method calls only of the test, user action and adapter layers, along with the arguments and the return values (if applicable).

\begin{table}
\caption{Formatted audited data report}
\rowcolors{2}{gray!25}{white}
\begin{tabular}{l l c}
\rowcolor{gray!50}
\hline\hline

Method call & Arguments & Returns \\ [0.5ex] % inserts table heading
\hline 

\hskip 0cm EditorTest.saveNewTextFile & & PASSED \\
\hskip 3mm   EditorUserActions.createNewTextFile & & \\
\hskip 6mm     MenuAdapter.clickFileMenu & & \\ 
\hskip 6mm     MenuAdapter.clickNewTextFileMenuItem & & \\ 
\hskip 3mm   EditorUserActions.enterText & text=''hello'' & \\
\hskip 6mm     TextAreaAdapter.getCaretPosition & & 0 \\
\hskip 6mm     TextAreaAdapter.insertText & position=0, text=''hello'' & \\
\hskip 3mm   EditorUserActions.enterText & text=''world'' & \\
\hskip 6mm     TextAreaAdapter.getCaretPosition & & 5 \\
\hskip 6mm     TextAreaAdapter.insertText & position=5, text='' world'' & \\
\hskip 3mm   EditorUserActions.saveCurrentEditor & filename=''C:{\textbackslash}helloWorld.txt'' & \\
\hskip 6mm     MenuAdapter.clickFileMenu & & \\
\hskip 6mm     MenuAdapter.clickSaveMenuItem & & \\
\hskip 6mm     DialogAdapter.setFileName & filename=''C:{\textbackslash}helloWorld.txt'' & \\
\hskip 6mm     DialogAdapter.clickSaveButton & & \\
\hskip 6mm     DialogAdapter.expectToDisappear & & \\	 
\hskip 3mm   EditorUserActions.expectTitle & title=''helloWorld'' & \\
\hskip 6mm     MainWindowAdapter.expectTitle & title=''My Notepad - helloWorld'' & \\
\hskip 6mm     FileUtils.expectFileContents &
    \begin{tabular}[x]{@{}c@{}}
       path=''C:{\textbackslash}helloWorld.txt", \\
      contents=''hello world''
    \end{tabular} & \\

\hline

\end{tabular}
\label{fig:formatted_audited_data_report}
\end{table}

\section{Automatic EDT dispatch}\label{sec:edt-dispatch}

\subsection{Cumbersome synchronization with EDT}\label{sec:edt-sync}

Since the threading model of Swing recommends that all Swing-component-related operations are to be made on the Swing EDT thread, including those of the tests, it is the test developer's responsibility to delegate all such work to the EDT thread by using FEST-Swing's GuiActionRunner. For tests using the User Action and Adapter design pattern, this can lead to much boilerplate code in the adapter layer. 

A solution to having to manually delegate work to EDT would be to mark the methods that needs to run on EDT with a certain annotation and let the runtime system do the delegation automatically. This would lead to lesser effort from the developer's side, and at the same time, would lead to increased security on the tests’ correctness.

There can also be methods that must not be invoked on EDT (e.g. most FEST methods, because they check the contents of event queues). Executing those methods on EDT can lead either to exceptions, to the tests hanging, or simply to incorrect behavior. Marking such methods with a certain annotation and letting the runtime system ensure that the test fails whenever the test executes such a method on EDT would also lead to increased security on the tests' correctness.

\subsection {Automatic EDT dispatch}\label{sec:edt-dispatch}

The operations need to be marked with certain annotations and the runtime system behind the scenes automatically handles the delegation.

\section {Simulated drag-and-drop}\label{sec:simulated-dnd}

\subsection{Defective DnD with Cacio-tta}\label{sec:defective-dnd}

FEST-Swing completely supports mouse gesture-based drag-and-drop (DnD) between Swing components, allowing the tests to cover that part of the application's functionality. However, some limitations (see above \fullref{sec:fest-swing-limitations}) often required the tests system to use the alternative graphics stack implementation named Cacio-tta \cite{Cacio_Tta_FEST} with the known limitation that it does not provide the required system-level support for drag and drop. Thus, in such test systems the mouse gesture-based approach of FEST-Swing's does not work.

A solution would be to programmatically trigger the data export and import operations on the Swing components, thus simulating the drag and drop without any mouse movement. However, because of the platform-dependent nature of the Swing (and AWT) DnD subsystem \cite{IntroDnD} \cite{DnDSubsystem}, this is not trivial to be realized.

\subsection{Drag and drop in Swing - Behind the scenes}\label{sec:dnd-swing}

Let us say that the user, who is running a Java application, wants to drag some text from a list and deposit it into a text field. Briefly, the drag and drop process consists of the following steps \cite{IntroDnD}:

\begin{enumerate}
\item The user has selected a row of text in the source component: the list. While holding the mouse button the user begins to drag the text — this initiates the drag gesture.
\item \label{item:drag-begin} When the drag begins, the list component packages up the data for export and declares what source actions it supports (i.e. COPY, MOVE, or LINK).
\item As the user drags the data, Swing continuously calculates the location and handles the rendering.
\item If the user simultaneously holds the Shift and/or Control key during the drag, this user action is also part of the drag gesture. Typically, an ordinary drag requests the MOVE action. Holding the Control key while dragging requests the COPY action, and holding both Shift and Control requests the LINK action.
\item \label{item:insert-location} Once the user drags the text over the bounds of a text field component, the target is continually polled to see if it will accept or reject the potential drop. As he drags, the target provides feedback by showing the insert location, perhaps an insertion cursor or a high-lighted selection. In the current example, the text field (the current target) allows both re-placement of selected text and insertion of new text.
\item \label{item:import-data} When the user releases the mouse button, the text component imports the data by inspecting the declared source actions and any user action and then it chooses what it wants out of the available options. In the current example, the text field chooses to insert the new text at the point of the drop.
\end{enumerate}

While this might seem like a daunting process, Swing handles most of the work out of the box. The Swing framework is designed so that the developer plug in the details specific to the components, and the rest is automatic.

\subsection {Simulated drag-and-drop}\label{sec:simulated-dnd}

By doing the following steps, the tests can programmatically trigger drag-and-drop:
\begin{enumerate}
\item Trigger an export on the source component (see point \ref{item:drag-begin} above).
\item Set the proper insert location at the destination component (see point \ref{item:insert-location} above).
\item Trigger the import of the data (see point \ref{item:import-data} above).
\end{enumerate}

\section{Non-functional requirements}\label{sec:nfr}

\subsection{Scala}\label{sec:nfr-scala}

The language of choice for implementing the extension framework should be Scala. It offers advanced functional programming paradigms for the JVM platform, while keeping full compatibility and interoperability with Java systems. \cite{ScalaWiki}

\subsection{Integration with Spring}\label{sec:nfr-spring}
The extension framework should support tests built on Spring Framework, \cite{SpringWiki} the popular enterprise application framework. At least the configuration of the extension framework should be possible through the Spring context, and some or all of the features could be based on Spring’s AOP services.

\subsection{Integration with Maven}\label{sec:nfr-maven}

The extension framework should support integration with Apache Maven, \cite{MavenWiki} the popular build automation tool.  The audit report should be integrated into the tests’ standard surefire or failsafe reports. If the AspectJ-based features of the extension framework need any additional pre-compilation steps or advanced classpath configuration, they should also be implemented by custom Maven plugins integrating into the corresponding Maven lifecycle phases.

\section{Scala Commander}\label{sec:scala-commander}

Scala Commander is an application developed to demonstrate the capabilities of FEST-Logging. It is a simple Swing-based file manager application developed in Scala, with a reactive user interface that was implemented using the Scala library \emph{Scala-React}. Scala-React is described by the article \emph{Deprecating the Observer Pattern with Scala.React}\cite{DeprecatingObservers}.















\appendix

\chapter{Scala CPS plugin examples}\label{chap:cps-examples}

Example \ref{fig:example_cps_1} shows a simple numeric example. The value of the \texttt{reset} block is equal to the value of the \texttt{shift} block plus one. The \texttt{shift}, block gets the rest of the continuation in the form of the function \texttt{k}, which in this case is the function \(x + 1\). The shift block invokes \texttt{k} with the argument 7 and returns the resulting value 8, which is then the value of the \texttt{reset} block. This value is then multiplied by 2 (outside the continuation), so the printed value is 16.

\begin{figure}[h!] \label{fig:example_cps_1}
\begin{lstlisting}
val v1 = reset {
  shift { k: (Int=>Int) =>
    k(7)
  } + 1
} * 2
println(v1) // prints 16
\end{lstlisting}
\caption{Simple example directly processing the continuation at its only \texttt{shift} block.}
\end{figure}

Example \ref{fig:example_cps_step_by_step} shows the control flow throughout the \texttt{entire} reset block. Notice the similarity with multiple function calls.

\begin{figure}[h!] \label{fig:example_cps_step_by_step}
\begin{lstlisting}
val result = reset {
  println("entering first shift")
  val firstShift = shift { k: (Int => Int) =>
      val res = k(0)
      println(s"exiting first shift, res = $res")
      res
  } + 1

  println(s"firstShift = $firstShift; entering second shift")
  val secondShift = shift { k: (Int => Int) =>
      val res: Int = k(firstShift)
      println(s"exiting second shift, res = $res")
      res
  } + 1
  println(s"secondShift = $secondShift; returning the reset")

  secondShift
}

println(s"result = $result")

// entering first shift
// firstShift = 1; entering second shift
// secondShift = 2; returning the reset
// exiting second shift, res = 2
// exiting first shift, res = 2
// result = 2
\end{lstlisting}
\caption{Example showing the control flow throughout the reset block when there is more than one shift block.}
\end{figure}

Example \ref{fig:example_cps_2} shows that the rest of the continuation may be called multiple times. In this case, the rest of the continuation is invoked three times. The final result of the continuation is multiplied by 2 outside the continuation (resulting in 20).

\begin{figure}[h!] \label{fig:example_cps_2}
\begin{lstlisting}
val v2 = reset {
  shift { k: (Int=>Int) =>
    k(k(k(7)))
  } + 1
} * 2
println(v2) // prints 20
\end{lstlisting}
\caption{Multiple invocation of the continuation in the \texttt{shift} block.}
\end{figure}





{ \renewcommand{\baselinestretch}{0.8}\normalsize %
  \setlength{\itemsep}{-2.4mm}
  \setlength{\bibspacing}{0.67\baselineskip}
  \bibliographystyle{abbrvnat}
  \bibliography{paper_bibliography}
}


\end{document}
