
\chapter{Scala CPS plugin examples}\label{chap:cps-examples}

Example \ref{fig:example_cps_1} shows a simple numeric example. The value of the \texttt{reset} block is equal to the value of the \texttt{shift} block plus one. The \texttt{shift}, block gets the rest of the continuation in the form of the function \texttt{k}, which in this case is the function \(x + 1\). The shift block invokes \texttt{k} with the argument 7 and returns the resulting value 8, which is then the value of the \texttt{reset} block. This value is then multiplied by 2 (outside the continuation), so the printed value is 16.

\begin{figure}[h!] \label{fig:example_cps_1}
\begin{lstlisting}
val v1 = reset {
  shift { k: (Int=>Int) =>
    k(7)
  } + 1
} * 2
println(v1) // prints 16
\end{lstlisting}
\caption{Simple example directly processing the continuation at its only \texttt{shift} block.}
\end{figure}

Example \ref{fig:example_cps_step_by_step} shows the control flow throughout the \texttt{entire} reset block. Notice the similarity with multiple function calls.

\begin{figure}[h!] \label{fig:example_cps_step_by_step}
\begin{lstlisting}
val result = reset {
  println("entering first shift")
  val firstShift = shift { k: (Int => Int) =>
      val res = k(0)
      println(s"exiting first shift, res = $res")
      res
  } + 1

  println(s"firstShift = $firstShift; entering second shift")
  val secondShift = shift { k: (Int => Int) =>
      val res: Int = k(firstShift)
      println(s"exiting second shift, res = $res")
      res
  } + 1
  println(s"secondShift = $secondShift; returning the reset")

  secondShift
}

println(s"result = $result")

// entering first shift
// firstShift = 1; entering second shift
// secondShift = 2; returning the reset
// exiting second shift, res = 2
// exiting first shift, res = 2
// result = 2
\end{lstlisting}
\caption{Example showing the control flow throughout the reset block when there is more than one shift block.}
\end{figure}

Example \ref{fig:example_cps_2} shows that the rest of the continuation may be called multiple times. In this case, the rest of the continuation is invoked three times. The final result of the continuation is multiplied by 2 outside the continuation (resulting in 20).

\begin{figure}[h!] \label{fig:example_cps_2}
\begin{lstlisting}
val v2 = reset {
  shift { k: (Int=>Int) =>
    k(k(k(7)))
  } + 1
} * 2
println(v2) // prints 20
\end{lstlisting}
\caption{Multiple invocation of the continuation in the \texttt{shift} block.}
\end{figure}


